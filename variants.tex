\documentclass[czech]{article}
\usepackage{graphicx} % Required for inserting images
\usepackage{babel}
\usepackage[a5paper,top=2cm,bottom=2cm,left=2cm,right=2cm]{geometry}
\usepackage{amsmath}
\usepackage{wrapfig}
\usepackage{tikz}
\usepackage{circuitikz}
\usepackage[style=iso-numeric]{biblatex}
\usepackage{csquotes}

\newcounter{variantcounter}


\newenvironment{variant}
{
    \vspace{0pt}
    \ifdim\pagetotal>\dimexpr\textheight-5\baselineskip\relax
        \newpage
    \fi
    % skupina pro to aby byl counter a text u sebe
    \begingroup
    \addtocounter{variantcounter}{1}
    \thevariantcounter{}.
}
{
    \endgroup
}

\begin{document}
\begin{center}
	\textbf{Možnosti pro vizualici ve VR}
\end{center}
\begin{large}
\textbf{Důležité pojmy}

\end{large}

\textit{VR} - Virtual Reality (virtuální realita)

\textit{AR} - Augmented Reality (rozšířená realita) - \enquote{přidání} digitální objektů do 3D skenu reality

\textit{MR} - Mixed Reality (smíšená realita) - spojení VR a AR pro projekci virtuálních vizualizací do reálného světa

\textit{XR} - eXtended Reality - širší pojem označující VR, AR či MR - chápeme jako např. \enquote{zelenina} a pod ní okurka, mrkev, ...

\textit{OpenVR standard} - SDK (software development kit) a API (application programming interface) pro podporu SteamVR a jiných VR brýlí (nejzajímavěji pro nás HTC Vive, Oculus Rift, Windows MR)



\begin{variant}
	\textbf{Unity}
		
	Je možno využít game engine Unity a jeho vestavěné moduly k práci ve virtuální realitě. Mezi jeho výhodami je například jednoduchá workflow, relativně velká komunita - hodně tutoriálů a dobrá dokumentace. Nevýhodou je, že není open source, tudíž je to takový \enquote{black box}.
		
	\textbf{Podporované platformy}: Apple visionOS, všechny podporované OpenVR standardem
	
	\textbf{Příklad aplikace}: Among Us VR, LEGO Builder's Journey
\end{variant}

	
	
\begin{variant}
	\textbf{Unreal Engine}
		
	Podobně jako Unity lze využít tento game engine k vizualizaci v XR. Nevýhodou je, podobně jako u Unity, že většina funkcí pro nás funguje jako \enquote{black box}.
		
	Podporované platformy: OpenVR standard, PSVR, ARCore
		
	\textbf{Příklad aplikace}: Precision OS simulace pro chirurgy, Toyota ergonomics evaluation
\end{variant}



\begin{variant}
	\textbf{NVIDIA IndeX}
		
	SDK pro vizualizaci a manipulaci s masivními datasety od NVIDIA. Dokáže vizualizovat obrovskou kvantitu dat (v ukázkách můžeme nalézt např. vizualizaci 150TB dat). Dobré pro vědecké aplikace.
\end{variant}



\begin{variant}
	\textbf{ParaView}
		
	ParaView podporuje XR k zobrazení a manipulaci dat v různých formátech (z nichž je pro nás nejzajímavější VTK). Tato možnost nám neumožní vytvořit vlastní aplikace, je to pouze vestavěný modul v již existující aplikaci, tudíž je pro nás pouze zajímavostí, ne \enquote{opravdovou} možností.
		
	\textbf{Podporované platformy}: Vše podporované OpenVR standardem
\end{variant}


\begin{variant}
	\textbf{OpenVR}
		
	SDK a API od Valve poskytující možnost manipulovat s VR hardware od různých 	výrobců bez potřeby toho aby aplikace věděla s jakým specifickým hardwarem pracuje. Výhodou je široká podpora hardwaru a dobrá dokumentace, také je vhodné pro náš dostupný hardware.
		
	\textbf{Příklad aplikace}: SteamVR, Half Life Alyx
\end{variant}



\end{document}
