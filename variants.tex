\documentclass[czech]{article}
\usepackage{graphicx} % Required for inserting images
\usepackage{babel}
\usepackage[a5paper,top=2cm,bottom=2cm,left=2cm,right=2cm]{geometry}
\usepackage{amsmath}
\usepackage{wrapfig}
\usepackage{tikz}
\usepackage{circuitikz}
\usepackage[style=iso-numeric]{biblatex}
\usepackage{csquotes}

\newcounter{variantcounter}

\newenvironment{variant}
{
	\addtocounter{variantcounter}{1}
	\thevariantcounter{}.
}


\begin{document}
\begin{center}
	\textbf{Možnosti pro vizualici ve VR}
\end{center}
\begin{large}
\textbf{Důležité pojmy}\\
\end{large}
\textit{VR} - Virtual Reality (virtuální realita)\\
\textit{AR} - Augmented Reality (rozšířená realita) - \enquote{přidání} digitální objektů do 3D skenu reality\\
\textit{MR} - Mixed Reality (smíšená realita) - spojení VR a AR pro projekci virtuálních vizualizací do reálného světa\\
\textit{XR} - eXtended Reality - širší pojem označující VR, AR či MR - chápeme jako např. \enquote{zelenina} a pod ní okurka, mrkev, ... \\
\textit{OpenVR standard} - SDK (software development kit) a API (application programming interface) pro podporu SteamVR a jiných VR brýlí (nejzajímavěji pro nás HTC Vive, Oculus Rift, Windows MR)\\
\begin{variant}
	\textbf{Unity}
	\par{}Je možno využít game engine Unity a jeho vestavěné moduly k práci ve virtuální realitě. Mezi jeho výhodami je například jednoduchá workflow, relativně velká komunita - hodně tutoriálů a dobrá dokumentace. Nevýhodou je, že není open source, tudíž je to takový \enquote{black box}.
	\par{}Podporované platformy: Apple visionOS, všechny podporované OpenVR standardem
	\par{}Příklad aplikace vytvořené pomocí této metody: Among Us VR, LEGO Builder's Journey
\end{variant}
\begin{variant}
	\textbf{ParaView}
	\par{}ParaView podporuje XR k zobrazení a manipulaci dat v různých formátech (z nichž je pro nás nejzajímavější VTK). Tato možnost nám neumožní vytvořit vlastní aplikace, je to pouze vestavěný modul v již existující aplikaci, tudíž je pro nás pouze zajímavostí, ne \enquote{opravdovou} možností.
	\par{}Podporované platformy: Vše podporované OpenVR standardem
\end{variant}
\begin{variant}
	\textbf{Unreal Engine}
	\par{}Podobně jako Unity lze využít tento game engine k vizualizaci v XR. 
\end{variant}
\end{document}