\documentclass[czech]{article}
\usepackage{graphicx} % Required for inserting images
\usepackage{babel}
\usepackage[a5paper,top=2cm,bottom=2cm,left=2cm,right=2cm]{geometry}
\usepackage{amsmath}
\usepackage{wrapfig}
\usepackage{tikz}
\usepackage{circuitikz}
\usepackage[style=iso-numeric]{biblatex}
\usepackage{csquotes}


\begin{document}
\section{Úvod}
\par{Virtuální realita (VR), rozšířená realita (AR) a smíšená realita (MR), dohromady známé jako rozšířená realita (XR) je rapidně rozvíjející se oblast, která nám umožňuje zpracovávat data v novém virtuálním prostředí. Tyto technologie nám poskytují platformu pro manipulaci s daty a objekty v 3D prostoru a čím dál častěji se v dnešní době používají v oblasti výzkumu, medicíny či zábavním průmyslu. Pro práci v XR můžeme využít řadu různých nástrojů, knihoven či už existujících aplikací.}
\par{První z možností, na které se podíváme, jsou game enginy s podporou pro XR, jako je například Unity či Unreal Engine. Tyto prostředí nám umožňují komfortně pracovat s různými již předpřipravenými knihovnami a pluginy pro vytváření XR aplikací. Omezením je podpora platforem a uzavřenost kódu dříve zmíněných knihoven a pluginů. Tyto enginy podporují pouze omezený počet XR hardware jako například HTC Vive nebo Oculus Rift.}
\par{Další možnost, kterou lze využít, jsou již existující aplikace, jako například NVIDIA IndeX a ParaView. NVIDIA IndeX umožňuje vizualizaci obrovských datasetů a je vhodný pro využití v oblasti vědy. ParaView nabízí možnost zpracovávat v XR data formátu VTK, což je pro vědecké účely taky zajímavé. Nevýhodou této varianty je, že to jsou pouze nástroje, zaměřené na specifickou předem danou činnost, tudíž je nemůžeme až tolik přizpůsobit naším potřebám.}
\par{Následující a pro nás z hlediska programátora nejzajímavější možnost je rovnou pracovat s knihovnou OpenVR. Tato knihovna od Valve nám umožňuje vytvářet multiplatformní XR aplikace s takovou flexibilitou, že pro nás hardware prakticky není překážka. Tato knihovna je velmi univerzální a také je open source, což je pro nás výhoda. Na této knihovně je postaveno nespočet aplikací, mezi nimiž najdeme například SteamVR.}
\par{Pro naše potřeby bude nejvhodnější práce rovnou s knihovnou OpenVR, jelikož je přístupná, dobře zdokumentovaná a má nativní podporu pro náš dostupný hardware.}


\end{document}