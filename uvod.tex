\documentclass[czech]{article}
\usepackage{graphicx} % Required for inserting images
\usepackage{babel}
\usepackage[a5paper,top=2cm,bottom=2cm,left=2cm,right=2cm]{geometry}
\usepackage{amsmath}
\usepackage{wrapfig}
\usepackage{tikz}
\usepackage{circuitikz}
\usepackage[style=iso-numeric,backend=bibtex]{biblatex}
\usepackage{csquotes}
\addbibresource{bibliografie.bib}

\begin{document}
\section{Úvod}
\par{Virtuální realita (VR), rozšířená realita (AR) a smíšená realita (MR), dohromady známé jako rozšířená realita (XR) je rapidně rozvíjející se oblast, která nám umožňuje zpracovávat data v novém virtuálním prostředí. Tyto technologie nám poskytují platformu pro manipulaci s daty a objekty v 3D prostoru a čím dál častěji se v dnešní době používají v oblasti výzkumu, medicíny či zábavním průmyslu. Pro práci v XR můžeme využít řadu různých nástrojů, knihoven či už existujících aplikací.}
\par{První z možností, které zmíníme, jsou game enginy Unity\textbf{ODCITOVAT} a Unreal Engine\textbf{ODCITOVAT}. Tyto enginy jsou využívány například ve stavebnictví s aplikací od firem OutHere a Skanska\textbf{ODCITOVAT}, ve zdravotnictví s produkty od Virtamed\textbf{ODCITOVAT} nebo Precision OS\textbf{ODCITOVAT} za účelem trénování zaměstnanců. Game enginy nám poskytují schopnost pracovat s již předpřipravenými nástroji, např. detekce inputu\textbf{ODCITOVAT} nebo nástroji pro multiplatformní vývoj\textbf{ODCITOVAT}.}
\par{Dále jsou na trhu aplikace ParaView\textbf{ODCITOVAT} a její rozšíření NVIDIA IndeX\textbf{ODCITOVAT}, které jsou v případě ParaView zaměřené na práci s formáty běžně nalezenými ve vědecké činnosti a v případě NVIDIA IndeX zaměřené na zpracování velkého počtu dat pomocí výpočetních clusterů\textbf{ODCITOVAT}. Tyto aplikace ale neposkytují možnost vytvořit další aplikace.}
\par{Následovně můžeme hovořit o knihovně OpenVR\textbf{ODCITOVAT}, což je open source knihovna pod licensí BSD-3-Clause license\textbf{ODCITOVAT}, která nám umožňuje vytvářet vlastní aplikace od úplného začátku. Tohle je výhodné v případě, že chceme hlouběji manipulovat s hardwarem, naopak nevýhodou je málo dokumentace.\textbf{ODCITOVAT - dokumentace ?}.}
\par{Hardware, který můžeme využít je např. Meta Quest\textbf{ODCITOVAT}, Valve Index\textbf{ODCITOVAT}, HTC Vive\textbf{ODCITOVAT}.}
\par{V této práci jsme se rozhodli pro použití ... a to z důvodů ...}
\par{.}


\newpage
\printbibliography
\end{document}